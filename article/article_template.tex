\documentclass[10 pt,a4paper]{article}

% packages 
\usepackage{amsthm} % Theorem Environment 
\usepackage{graphicx}
\usepackage{tikz}
\usepackage{lmodern}
\usepackage{booktabs}
\usepackage{multirow}
\usepackage{tabularx}
\usepackage{natbib}
\usepackage{pifont}
\usepackage{geometry}
\usepackage{authblk}
\usepackage{amsmath}
\usepackage[title]{appendix} % Add ``Appendix'' before section titles
\usepackage{titlesec} % to change font size of section titles

% Hyperlinks
\usepackage{hyperref}
\hypersetup{
     colorlinks = true,
     citecolor = red!60!black,
     linkcolor = red!60!black  
 } 

% Fonts
\usepackage{mathpazo} % Palatino font for math and text
\linespread{1.05} % Palatino needs more space between lines

% Whitespace  
\geometry{a4paper, top=25mm, left=30mm, right=25mm, bottom=30mm,headsep=10mm, footskip=12mm}

% Font size of section titles
\titleformat*{\section}{\large\bfseries}
\titleformat*{\subsection}{\normalsize\bfseries}

% Macros
% Theorem, Lemma, Corollary, Definition
\newtheorem{theorem}{Theorem}[section]
\newtheorem{corollary}{Corollary}[theorem]
\newtheorem{lemma}[theorem]{Lemma}
\newtheorem{definition}{Definition}

% Economics
\newtheorem*{household}{Household Problem}
\newtheorem*{firm}{Firm Problem}

% Statistics
\newcommand\pN{\mathcal{N}} % Normal distribution
\newcommand\iidN{\overset{iid}{\sim}\mathcal{N}} % iid Normal
\newcommand\E{\mathbb{E}} % Expectations Operator



% Title page
\title{\LARGE Latex Article Template}
\date{}

\author[1]{Author A}
\author[2]{Author A}
\author[3]{Author A}

\affil[1]{\it\small Department of Economics, University 1, ...}
\affil[2]{\it\small Department of Statistics, University 2, ...}
\affil[3]{\it\small Department of Mathematics, University 3, ...}

\begin{document}

% Title and abstract
\maketitle
\thispagestyle{empty}

\par\noindent\rule{\textwidth}{0.4pt} % Line before abstract
\begin{abstract}
Adam Smith was a Scottish economist and philosopher and an important figure during the Scottish Enlightenment era. Smith is best known for his book ``An Inquiry into the Nature and Causes of the Wealth of Nations''. This article cites a passage from that book.
\end{abstract}
\par\noindent\rule{\textwidth}{0.4pt} % Line after abstract
\vspace*{-1.7cm} % control whitespace after line

\section*{} % Do not show the title for the introduction 

This article cites a passage from \citet{smith1827inquiry}.
 
\section{The Passage}

As every individual, therefore, endeavours as much as he can both to employ his capital in the support of domestic industry, and so to direct that industry that its produce may be of the greatest value; every individual necessarily labours to render the annual revenue of the society as great as he can. He generally, indeed, neither intends to promote the public interest, nor knows how much he is promoting it. By preferring the support of domestic to that of foreign industry, he intends only his own security; and by directing that industry in such a manner as its produce may be of the greatest value, he intends only his own gain, and he is in this, as in many other cases, led by an invisible hand to promote an end which was no part of his intention. Nor is it always the worse for the society that it was no part of it. By pursuing his own interest he frequently promotes that of the society more effectually than when he really intends to promote it.

\bibliographystyle{plainnat}
\bibliography{article_template}
\end{document}



